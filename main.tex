\documentclass{article}

\usepackage{morewrites}
\usepackage[pdf]{graphviz}
\usepackage{ragged2e}
\usepackage[pdftex]{graphicx}
\usepackage[utf8]{inputenc}			% кодировка исходного текста
\usepackage[english,russian]{babel}	% локализация и переносы
% Математика
\usepackage{amsmath,amsfonts,amssymb,amsthm,mathtools,amstext} 
\usepackage{amsmath}

\begin{document}

    \begin{titlepage}

        \thispagestyle{empty}

        \centerline{НАЦИОНАЛЬНЫЙ ИССЛЕДОВАТЕЛЬСКИЙ УНИВЕРСИТЕТ}
        \centerline{МОСКОВСКИЙ ЭНЕРГЕТИЧЕСКИЙ ИНСТИТУТ}

        \vfill

        \centerline{\huge{Домашняя работа}}
        \centerline{\large{по дисциплине}}
        \centerline{\LARGE{Теоретические модели вычислений}}

        \vfill

        \largeСтудент группы А-13б-19 \hfill \largeХрущев А.П.

        \largeПреподаватель \hfill \largeИвлиев С.А.

        \vfill

        \centerline{Москва, 2022}
        \clearpage
    \end{titlepage}

    \section*{\huge{\centerline{Задание №1.} \\Построить конечный автомат, распознающий язык.}}
        \begin{enumerate}
            
            \LARGE
            \item \(L = \{ \omega \in \{a,b,c\}^*| \ |\omega|_c=1 \}\) \\
            
            \begin{center}
                \digraph{1.1.1} 
            \end{center}
            
             \item \(L = \{ \omega \in \{a,b\}^*| \ |\omega|_a \leq 2, |\omega|_b \geq 2 \}\) \\

            \begin{center}
                \digraph{1.2.1} 
            \end{center}

            \item \(L = \{ \omega \in \{a,b\}^*| \ |\omega|_a \neq |\omega|_b \}\) \\
            
            Докажем для \( \overline L\) \\
            \(\omega = a^{n}b^{n}\), 
        	\(x = b^{i}\) \\
        	\(y = b^{j}\)  \\
        	\(z = b^{n-i-j}a^{n}\) \\
        	\(\omega = xyz = b^{i}b^{j*k}b^{n-i-j}a^{n} = b^{n-j*(k-1)}a^{n}\notin L\)
        	\(\implies\)не регулярный
        	
            \item \(L = \{ \omega \in \{a,b\}^*| \ | \omega\omega = \omega\omega\omega \}\) \\

            \begin{center}
                \digraph{1.4.1} 
            \end{center}
        
        \end{enumerate}
    
    \section*{\huge{\centerline{Задание №2.} \\Постоить конечный автомат, используя прямое произведение.}}
        \begin{enumerate}
            \LARGE
            \item \( L_1 = \{ \omega \in \{a,b\}^*| \ |\omega|_a \geq 2 \land |\omega|_b \geq 2 \}\) \\
            
            \(A = \langle{Q_a, \sum_a, S_a, T_a, \sigma_a}\rangle\) \\
            \\
            \hspace*{-20mm}\(Q_a = \{ q_1, q_2, q_3 \};\) \\
            \hspace*{-20mm}\(\sum_a = \{ a,b \};\) \\
            \hspace*{-20mm}\(S_a = q_1;\) \\
            \hspace*{-20mm}\(T_a = q_3;\)
            
            \begin{center}
                \digraph{2.1.1} 
            \end{center}
            
            \(B = \langle{Q_b, \sum_b, S_b, T_b, \sigma_b}\rangle\) \\
            \\
            \hspace*{-20mm}\(Q_b = \{ q_4, q_5, q_6 \};\) \\
            \hspace*{-20mm}\(\sum_b = \{ a,b \};\) \\
            \hspace*{-20mm}\(S_b = q_4;\) \\
            \hspace*{-20mm}\(T_b = q_6;\)
            
            \begin{center}
                \digraph{2.1.2} 
            \end{center}
            
            \(A\ \times\  B = \langle{\sum_a\cup\sum_b,\ Q_a\times Q_b,\ \langle S_a,\ S_b\rangle,\ T_a\times T_b,\ \langle \sigma_a,\ \sigma_b\rangle }\rangle\)\\
            \\
            \hspace*{-20mm}\(\sum = \{ a,b \};\) \\
            \hspace*{-20mm}\(Q = \{\langle q_1,q_4\rangle, \langle q_1,q_5\rangle, \langle q_1,q_6\rangle, \langle q_2,q_4\rangle, \langle q_2,q_5\rangle, \langle q_2,q_6\rangle, \)\\
            \hspace*{55mm}\(\langle q_3,q_4\rangle, \langle q_3,q_5\rangle, \langle q_3,q_6\rangle\};\)\\
            \hspace*{-20mm}\(S = \langle q_1,q_4\rangle;\)\\
            \hspace*{-20mm}\(T = \langle q_3,q_6\rangle;\)\\
            
            \centerline{Таблица состояний:}
            
            \begin{tabular}{ | l | l | l | }
                \hline
                Состояние & По a & По b \\ \hline
                \(\langle q_1, q_4\rangle\) & \(\langle q_1, q_5\rangle\) & \(\langle q_2, q_4\rangle\)  \\ \hline
                \(\langle q_1, q_5\rangle\) & \(\langle q_1, q_6\rangle\) & \(\langle q_2, q_5\rangle\) \\ \hline
                \(\langle q_2, q_4\rangle\) & \(\langle q_2, q_5\rangle\) & \(\langle q_3, q_4\rangle\) \\ \hline
                \(\langle q_2, q_5\rangle\) & \(\langle q_2, q_6\rangle\) & \(\langle q_3, q_5\rangle\) \\ \hline
                \(\langle q_1, q_6\rangle\) & \(\langle q_1, q_6\rangle\) & \(\langle q_2, q_6\rangle\) \\ \hline
                \(\langle q_3, q_4\rangle\) & \(\langle q_3, q_5\rangle\) & \(\langle q_3, q_4\rangle\) \\ \hline 
                \(\langle q_2, q_6\rangle\) & \(\langle q_2, q_6\rangle\) & \(\langle q_3, q_6\rangle\) \\ \hline
                \(\langle q_3, q_5\rangle\) & \(\langle q_3, q_6\rangle\) & \(\langle q_3, q_5\rangle\) \\ \hline
                \(\langle q_3, q_6\rangle\) & \(\langle q_3, q_6\rangle\) & \(\langle q_3, q_6\rangle\) \\
            \hline
            
            \end{tabular} \\
            
            \begin{center}
                \digraph{2.1.3} 
            \end{center}
            
            \LARGE
            \item \( L_2 = \{ \omega \in \{a,b\}^*| \ |\omega| \geq 3 \land |\omega| \  \text{нечётное} \}\) \\
            
            \(A = \langle{Q_a, \sum_a, S_a, T_a, \sigma_a}\rangle\) \\
            \\
            \hspace*{-20mm}\(\sum_a = \{ a,b \}\) \\
            \hspace*{-20mm}\(Q_a = \{ q_1, q_2, q_3, q_4 \}\) \\
            \hspace*{-20mm}\(S_a = q_1\) \\
            \hspace*{-20mm}\(T_a = q_4\) \\
            
            \begin{center}
                \digraph{2.2.1} 
            \end{center}
            
            \(B = \langle{Q_b, \sum_b, S_b, T_b, \sigma_b}\rangle\) \\

            \hspace*{-20mm}\(\sum_b = \{ a,b \}\) \\
            \hspace*{-20mm}\(Q_b = \{ q_5, q_6\} \) \\
            \hspace*{-20mm}\(S_b = c_5\) \\
            \hspace*{-20mm}\(T_b = c_6\) \\

            \begin{center}
                \digraph{2.2.2} 
            \end{center}
            
            \(A\ \times\  B = \langle{\sum_a\cup\sum_b,\ Q_a\times Q_b,\ \langle S_a,\ S_b\rangle,\ T_a\times T_b,\ \\
            \langle \sigma_a,\ \sigma_b\rangle }\rangle\)\\
            
            \hspace*{-20mm}\(Q = \{\langle q_1,q_5\rangle, \langle q_1,q_6\rangle, \langle q_2,q_5\rangle, \langle q_2,q_6\rangle, \langle q_3,q_5\rangle, \langle q_3,q_6\rangle, \langle q_4,q_5\rangle, \langle q_4,q_6\rangle\};\)\\
            \hspace*{-20mm}\(\sum = \{ a,b \}\) \\
            \hspace*{-20mm}\(S = \langle q_1,q_5\rangle\) \\
            \hspace*{-20mm}\(T = \langle q_4,q_6\rangle\) \\

            \centerline{Таблица состояний:}
            
            \begin{tabular}{ | l | l | l | }
            \hline
            Состояние & По a & По b \\ \hline
            \(\langle q_1, c_5\rangle\) & \(\langle q_2, q_6\rangle\) & \(\langle q_2, q_6\rangle\)  \\ \hline
            \(\langle q_2, q_6\rangle\) & \(\langle q_3, q_5\rangle\) & \(\langle q_3, q_5\rangle\) \\ \hline
            \(\langle q_3, q_5\rangle\) & \(\langle q_4, q_6\rangle\) & \(\langle q_4, q_6\rangle\) \\ \hline
            \(\langle q_4, q_6\rangle\) & \(\langle q_4, q_5\rangle\) & \(\langle q_4, q_5\rangle\) \\ \hline
            \(\langle q_4, q_5\rangle\) & \(\langle q_4, q_6\rangle\) & \(\langle q_4, q_6\rangle\) \\
            \hline
            \end{tabular} \\
            
            \begin{center}
                \digraph{2.2.3} 
            \end{center}
            
            \item \( L_3 = \{ \omega \in \{a,b\}^*| \ |\omega|_a \ \text{чётно} \  \land |\omega|_b \  \text{кратно трём} \}\) \\
        
            \(A = \{ \omega \in \{a,b\}^*| \ |\omega|_a \ \text{чётно} \}\)\\
            \\
            \hspace*{-20mm}\(\sum_1 = \{ a,b \}\) \\
            \hspace*{-20mm}\(Q_1 = \{ q_1, q_2 \}\) \\
            \hspace*{-20mm}\(S_1 = q_1\) \\
            \hspace*{-20mm}\(T_1 = q_1\) \\
            
            \begin{center}
                \digraph{2.3.1} 
            \end{center}

            \(B = \{ \omega \in \{a,b\}^*| \ |\omega|_b \  \text{кратно трём} \}\) \\
        
            \hspace*{-20mm}\(\sum_2 = \{ a,b \}\) \\
            \hspace*{-20mm}\(Q_2 = \{ q_3, q_4, q_5 \}\) \\
            \hspace*{-20mm}\(S_2 = q_3\) \\
            \hspace*{-20mm}\(T_2 = q_3\) \\
            
            \begin{center}
                \digraph{2.3.2} 
            \end{center}
            
            \(A\ \times\  B = \langle{\sum_a\cup\sum_b,\ Q_a\times Q_b,\ \langle S_a,\ S_b\rangle,\ T_a\times T_b,\ \\
            \langle \sigma_a,\ \sigma_b\rangle }\rangle\)\\
            \hspace*{-20mm}\(\sum = \{ a,b \}\) \\
            \hspace*{-20mm}\(Q = \{\langle q_1, q_3\rangle, \langle q_1, q_4\rangle, \langle q_1, q_5\rangle, \langle q_2, q_3\rangle, \langle q_2, q_4\rangle, \langle q_2, q_5\rangle\}\) \\
            \hspace*{-20mm}\(S = \langle q_1, q_3\rangle\) \\
            \hspace*{-20mm}\(T = \langle q_1, q_3\rangle\) \\
        
            \centerline{Таблица состояний:}
            
            \begin{tabular} { | l | l | l | }
            \hline 
            Состояние & По a & По b \\ \hline
            \(\langle q_1, q_3\rangle\) & \(\langle q_2, q_3\rangle\) & \(\langle q_1, q_4\rangle\)  \\ \hline
            \(\langle q_2, q_3\rangle\) & \(\langle q_1, q_3\rangle\) & \(\langle q_2, q_2\rangle\) \\ \hline
            \(\langle q_1, q_4\rangle\) & \(\langle q_2, q_4\rangle\) & \(\langle q_1, q_5\rangle\) \\ \hline
            \(\langle q_2, q_4\rangle\) & \(\langle q_1, q_4\rangle\) & \(\langle q_2, q_5\rangle\) \\ \hline
            \(\langle q_1, q_5\rangle\) & \(\langle q_2, q_5\rangle\) & \(\langle q_1, q_3\rangle\) \\ \hline
            \(\langle q_2, q_5\rangle\) & \(\langle q_1, q_5\rangle\) & \(\langle q_2, q_3\rangle\) \\
            \hline
            \end{tabular} \\
            
            \begin{center}
                \digraph{2.3.3} 
            \end{center}
            
            \item \( L_4 = \bar{L_3}\) \\
            \\
            \(\bar{L_3} = \{ \sum, Q, S, T, \delta  \}\) \\
            \hspace*{-20mm}\(\sum = \{a,b\}\)\\
            \hspace*{-20mm}\(Q = \{q_1,q_2,q_3,q_4,q_5,q_6\}\)\\
            \hspace*{-20mm}\(S = q_1\)\\
            \hspace*{-20mm}\(T = \{q_2,q_3,q_4,q_5,q_6\}\)\\
            
            \begin{center}
                \digraph{2.4.1}
            \end{center}
            
            
            \item \( L_5 = \frac{L_2}{L_3}\)  \\ 
            Автомат А:\\
            \begin{center}
                \digraph{2.5.1} 
            \end{center}
            
            Автомат Б:\\
            
            \digraph{2.4.1}
            
            \(\frac{L_2}{L_3} = L_2 \cap \bar{L_3} = L_2 \cap L_4\) \\ 
            \\
            
            \hspace*{-20mm}\(\sum = \{ a,b \}\) \\
            \hspace*{-20mm}\(Q = \{ \langle q_1, q_6 \rangle , \langle q_1, q_7\rangle , \langle q_1, q_8\rangle ,\langle q_1, q_9\rangle ,\langle q_1, q_{10}\rangle , \langle q_1, q_{11}\rangle ,\\ \hspace*{-5mm}\langle q_2, q_6 \rangle , \langle q_2, q_7\rangle , \langle q_2, q_8\rangle ,\langle q_2 ,q_9\rangle ,\langle q_2, q_{10}\rangle , \langle q_2, q_{11}\rangle , \\ \hspace*{-5mm}\langle q_3, q_6 \rangle , \langle q_3, q_7\rangle , \langle q_3, q_8\rangle ,\langle q_3, q_9\rangle ,\langle q_3, q_{10}\rangle , \langle q_3, q_{11}\rangle , \\ \hspace*{-5mm}\langle q_4, q_6 \rangle , \langle q_4, q_7\rangle , \langle q_4, q_8\rangle ,\langle q_4, q_9\rangle ,\langle q_4, q_{10}\rangle , \langle q_4, q_{11}\rangle , \\ \hspace*{-5mm}\langle q_5, q_6 \rangle , \langle q_5, q_7\rangle , \langle q_5, q_8\rangle ,\langle q_5, q_9\rangle ,\langle q_5, q_{10}\rangle , \langle q_5, q_{11}\rangle\}\) \\
            \hspace*{-20mm}\(S = \langle q_1, q_6\rangle\) \\
            \hspace*{-20mm}\(T = \{\langle q_4, q_7\rangle \langle q_4, q_8\rangle \langle q_4, q_9\rangle \langle q_4, q_{10}\rangle \langle q_4, q_{11}\rangle\}\) \\
            
            \begin{tabular} { | l | l | l | }
                \hline 
                Состояние & По a & По b \\ \hline
                \(\langle q_1, q_7\rangle\) & \(\langle q_2, q_8\rangle\) & \(\langle q_2, q_{10}\rangle\) \\ \hline
                \(\langle q_2, q_8\rangle\) & \(\langle q_3, q_7\rangle\) & \(\langle q_3, q_9\rangle\) \\ \hline
                \(\langle q_2, q_{10}\rangle\) & \(\langle q_3, q_9\rangle\) & \(\langle q_3, q_{11}\rangle\) \\ \hline
                \(\langle q_3, q_7\rangle\) & \(\langle q_4, q_8\rangle\) & \(\langle q_4, q_{10}\rangle\) \\ \hline
                \(\langle q_3, q_8\rangle\) & \(\langle q_4, q_{10}\rangle\) & \(\langle q_4, q_{11}\rangle\) \\ \hline
                \(\langle q_3, q_{11}\rangle\) & \(\langle q_4, q_{12}\rangle\) & \(\langle q_4, q_7\rangle\) \\ \hline
                \(\langle q_4, q_7 \rangle\) & \(\langle q_5, q_8\rangle\) & \(\langle q_5, q_{10}\rangle\) \\ \hline
                \(\langle q_4, q_8\rangle\) & \(\langle q_5, q_7\rangle\) & \(\langle q_5, q_9\rangle\) \\ \hline
                \(\langle q_4, q_9\rangle\) & \(\langle q_5, q_{10}\rangle\) & \(\langle q_5, q_{12}\rangle\) \\ \hline
                \(\langle q_4, q_{10}\rangle\) & \(\langle q_5, q_9\rangle\) & \(\langle q_5, q_{11}\rangle\) \\ \hline
                \(\langle q_4, q_{11}\rangle\) & \(\langle q_5, q_{12}\rangle\) & \(\langle q_5, q_7\rangle\) \\ \hline
                \(\langle q_4, q_{12}\rangle\) & \(\langle q_5, q_{11}\rangle\) & \(\langle q_5, q_8\rangle\) \\ \hline
                \(\langle q_5, q_7\rangle\) & \(\langle q_4, q_8\rangle\) & \(\langle q_4, q_{10}\rangle\) \\ \hline
                \(\langle q_5, q_8\rangle\) & \(\langle q_4, q_7\rangle\) & \(\langle q_4, q_9\rangle\) \\ \hline
                \(\langle q_5, q_9\rangle\) & \(\langle q_4, q_{10}\rangle\) & \(\langle q_4, q_{12}\rangle\) \\ \hline
                \(\langle q_5, q_{10}\rangle\) & \(\langle q_4, q_9\rangle\) & \(\langle q_4, q_{11}\rangle\) \\ \hline
                \(\langle q_5, q_{11}\rangle\) & \(\langle q_4, q_{12}\rangle\) & \(\langle q_4, q_7\rangle\) \\ \hline
                \(\langle q_5, q_{12}\rangle\) & \(\langle q_4, q_{11}\rangle\) & \(\langle q_4, q_8\rangle\) \\
                \hline
            \end{tabular} \\
            
            \begin{center}
                \digraph{2.5.3} 
            \end{center}
            
        \end{enumerate}
        
        \section*{\huge{\centerline{Задание №3.}\\ Постоить минимальный ДКА, по регулярному выражению.}}
        \begin{enumerate} 
            \LARGE
            \item \((ab+aba)^*a \) \\
            \\
            \hspace*{-20mm}{Построим НКА.}\\
            \digraph{3.1.1}
            \hspace*{-20mm}{Преобразуем в минимальный ДКА.}\\
            \begin{tabular} { | l | l | l | }
                \hline 
                Состояние & По a & По b \\ \hline
                \(q_1\) & \( q_{12}, q_5, q_6\) & \(-\) \\ \hline
                \(q_{12}, q_5, q_6\) & \(-\) & \( q_7, q_8\) \\ \hline
                \(q_7, q_8\) & \( q_{12}, q_5, q_6, q_9\) & \(-\) \\ \hline
                \(q_{12}, q_5, q_6, q_9\) & \( q_{12}, q_5, q_6\) & \(q_7, q_8\) \\ 
                \hline
            \end{tabular} \\
            \digraph{3.1.2}
            \LARGE
            \item \(a(a(ab(^*b)^*(ab)^* \) \\
            \\
            \hspace*{-20mm}{Построим НКА.}\\
            \digraph{3.2.1}
            \hspace*{-20mm}{Преобразуем в минимальный ДКА.}\\
            \begin{tabular} { | l | l | l | }
                \hline 
                Состояние & По a & По b \\ \hline
                \(q_1\) & \( q_2, q_6, q_8\) & \(-\) \\ \hline
                \(q_2, q_6, q_8\) & \(q_3, q_5, q_7, q_9\) & \( -\) \\ \hline
                \(q_3, q_5, q_7, q_9\) & \( q_4\) & \(q_2, q_6, q_8\) \\ \hline
                \(q_4\) & \( -\) & \(q_5, q_3\) \\ \hline
                \(q_5, q_3\) & \( q_4\) & \(q_2, q_6, q_8\) \\ 
                \hline
            \end{tabular} \\
            \digraph{3.2.2}
            \LARGE
            \item \((a+(a+b)(a+b)b)^* \) \\
            \\
            \hspace*{-20mm}{Построим НКА.}\\
            \digraph{3.3.1}
            \hspace*{-20mm}{Преобразуем в минимальный ДКА.}\\
            \begin{tabular} { | l | l | l | }
                \hline 
                Состояние & По a & По b \\ \hline
                \(q_1\) & \( q_5, q_9\) & \(q_5\) \\ \hline
                \(q_5, q_9\) & \(q_5, q_9, q_7\) & \( q_5, q_7\) \\ \hline
                \(q_5\) & \( q_7\) & \(q_7\) \\ \hline
                \(q_5, q_9, q_7\) & \( q_5, q_9, q_7\) & \(q_5, q_9, q_7\) \\ \hline
                \(q_5, q_5\) & \( q_7\) & \(q_7, q_9\) \\ \hline
                \(q_7\) & \( -\) & \(q_9\) \\ \hline
                \(q_7, q_9\) & \( q_5, q_9\) & \(q_5, q_9\) \\ \hline
                \(q_9\) & \( q_5, q_9\) & \(q_5\) \\ 
                \hline
            \end{tabular} \\
            \digraph{3.3.2}
            \LARGE
            \item \((b+c)((ab)^*c+(ba)^*)^* \) \\
            \\
            \hspace*{-20mm}{Построим НКА.}\\
            \digraph{3.4.1}
            \hspace*{-20mm}{Преобразуем в минимальный ДКА.}\\
            \begin{tabular} { | l | l | l | l | }
                \hline 
                Состояние & По a & По b & По c \\ \hline
                \(q_1\) & \( -\) & \(q_2\) & \(q_2\) \\ \hline
                \(q_2\) & \( q_4\) & \(q_7\) & \(-\) \\ \hline
                \(q_7\) & \( q_2\) & \(-\) & \(-\) \\ \hline
                \(q_4\) & \( -\) & \(q_5\) & \(-\) \\ \hline
                \(q_5\) & \( q_4\) & \(-\) & \(q_2\) \\
                \hline
            \end{tabular} \\
            \digraph{3.4.2}
            \LARGE
            \item \((a+b){ }^+(aa+bb+abab+baba)(a+b){ }^+   \) \\
            \\
            \hspace*{-20mm}{Построим НКА.}\\
            \digraph{3.5.1}
            \hspace*{-20mm}{Преобразуем в минимальный ДКА.}\\
            \digraph{3.5.2}
        \end{enumerate}
        \section*{\huge{\centerline{Задание №4.} \\Определить является ли язык регулярным.}}
        \begin{enumerate}
            \item\(L = \{ (aab)^n b(aba)^m | \ n \geq 0, m \geq 0 \}\) \\
            
            Существует ДКА, распазноющий данный язык, следовательно язык явлюется регулярным.\\
            \digraph{4.1.1}
            
            \item\(L = \{ uaav \ | \ u \in \{a,b\}^*, v \in \{a,b\}^*, |u|_b \geq |v|_a \}\)\\
            
            Фиксируем произвольное n. \\
	        \(\omega = a^{n}b^{n} |\omega|_b \geq n\), 
        	\(x = b^{i}\) \\
        	\(y = b^{j}\)  \\
        	\(z = b^{n-i-j}a^{n+2}\) \\
        	\(\omega = xy^0z = b^{i}b^{j}b^{n-i-j}a^{n+2} = b^{n-j}a^{n+2}\notin L\)
        	\(\implies\)не регулярный
            \item\(L = \{ a^m \omega \ | \ \omega \in \{a,b\}^*, 1 \leq |\omega|_b \leq m \}\)\\
        	Фиксируем произвольное n. \\
	        \(\omega = b^{n}aaa^{n} |\omega|_b \geq n\), 
        	\(x = a^{i}\) \\
        	\(y = a^{j}\)  \\
        	\(z = a^{n-i-j}b^{n}\) \\
        	\(\omega = xy^0z = a^{i}a^{0}a^{n-i-j}b^{n} = a^{n-j}b^{n} \notin L\)
        	\(\implies\)не регулярный
            \item\(L = \{ a^k b^m a^n \ | \ k = n \vee m > 0\) \} \\
            Существует ДКА, распазноющий данный язык, следовательно язык явлюется регулярным.\\
            \digraph{4.4.1}
            \item\(L = \{ucv \ | \ u \in \{a,b\}^*, v \in \{a,b\}^*, u \neq v^R \}\) \\
            Фиксируем произвольное \(n\). \\
        	\(\omega = (ab)^n c (ba)^n = a_1a_2...a_{4n+1}, \ |w| \geq n\)\\
        	\(\omega = xyz\)\\
        	\(x = a_1a_2...a_i\)\\
        	\(y = a_{i+1}a_{i+2}...a_{i+j}\)
        	\(z = a_{i+j+1}a{i+j+2}...a_{2n}c(ba)^n\)\\
	        \(xy^kz = a_1...a_i(a_{i+1}...a_{i+j})^k a_{i+j+1}..a_{2n}c(ba)^n \notin \bar L \ \forall k > 1\) не является регулярным \(\implies L\) не является регулярным. 
	       
\end{center}
        \end{enumerate}
\end{document}